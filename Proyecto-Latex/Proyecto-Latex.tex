\documentclass[10pt,a4paper]{article}
\usepackage[utf8]{inputenc}
\usepackage{amsmath}
\usepackage{amsfonts}
\usepackage{amssymb}
\usepackage{graphicx}
\begin{document}
\begin{center}
\textbf{Universidad de Guayaquil}

\textbf{Facultad de Ciencias Matematicas y Fisicas}

Proyecto de Implantación de un Sistema de Gestión de la Calidad ISO 9001:2015 en el Software SINFIG

\textbf{Integrantes:}

Ramirez Abarca Alex

Ramirez Rios Edinson 

Nicole Navarrete Briones

Curso: SOF-S-MA-3-2
\end{center}
\vspace{\baselineskip}
\textbf{Tabla de Contenido}
\vspace{\baselineskip}

\textbf{1. Normas ISO--------------------------------------------------------------}
\vspace{\baselineskip}

\textbf{2. Caso de Estudio-----------------------------------------------------------}
\vspace{\baselineskip}

\textbf{3. Planificación---------------------------------------------------------------}
\vspace{\baselineskip}

\textbf{4. Código Fuente-------------------------------------------------------------}
\vspace{\baselineskip}

\textbf{5. Conclusión-----------------------------------------------------------------}
\vspace{\baselineskip}

\textbf{1. NORMAS ISO}
\vspace{\baselineskip}

Las normas internacionales hacen que las cosas funcionen. Proporcionan especificaciones de clase mundial para productos, servicios y sistemas, para garantizar la calidad, seguridad y eficiencia. Son fundamentales para facilitar el comercio internacional.
Normas ISO 9001
ISO 9001: 2015 especifica los requisitos para un sistema de gestión de calidad cuando una organización:
a) necesita demostrar su capacidad para proporcionar constantemente productos y servicios que cumplan con los requisitos legales y reglamentarios aplicables del cliente, y
b) tiene como objetivo mejorar la satisfacción del cliente a través de la aplicación efectiva del sistema, incluidos los procesos para la mejora del sistema y el aseguramiento de la conformidad con el cliente y los requisitos legales y reglamentarios aplicables.
Todos los requisitos de ISO 9001: 2015 son genéricos y están destinados a ser aplicables a cualquier organización, independientemente de su tipo o tamaño, o los productos y servicios que proporciona.


\vspace{\baselineskip}
\vspace{\baselineskip}
\textbf{2. CASO DE ESTUDIO}
\vspace{\baselineskip}

Sinfig es un paquete de animación 2D basado en vectores. Está diseñado para ser capaz de producir animaciones con calidad de largometraje. Elimina la necesidad de interpolación, evitando la necesidad de dibujar a mano cada cuadro. Synfig presenta independencia de resolución espacial y temporal (nítida y suave a cualquier resolución o velocidad de fotogramas), imágenes de alto rango dinámico y un sistema de plugin flexible. Synfigstudio es el estudio de animación para synfig y proporciona la GUI interfaz para crear animaciones synfig que se guardan en synfig .sif o formato .sifz.
\vspace{\baselineskip}

\textbf{Características}

\begin{itemize}
\item Produce animaciones de calidad cinematográficas 
\item Mantiene una resolución independiente mente del territorio
\item Tiene una independencia de resolución temporal 
\item Es orientada a un diseño de Artista
\item Las animaciones se trabajan por capas 

\end{itemize}
\end{document}